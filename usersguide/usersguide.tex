\documentclass[11pt,oneside,a4paper]{memoir}
\usepackage{fontspec}
%\usepackage{showframe}
\usepackage{alltt}
\usepackage{xcolor}
\usepackage{tabu}
\usepackage{graphicx}
\usepackage{longtable}
\usepackage{multirow}
\usepackage{capt-of}
%\usepackage{wrapfig}
\usepackage[final]{listings}
\usepackage[unicode=true,xetex,colorlinks=true,linkcolor=blue,urlcolor=blue,bookmarksnumbered=true,bookmarksdepth=3]{hyperref}
\usepackage{bidi} % Must be last



%%%%%%%%%%%%%%%%%%%%%%%%%%%%%%%%%%%%%%%%%%%%%%%%%%%%%%%
%%%%%%%%%%%%%%%%%%%% Configuration %%%%%%%%%%%%%%%%%%%%
%%%%%%%%%%%%%%%%%%%%%%%%%%%%%%%%%%%%%%%%%%%%%%%%%%%%%%%

%%% Fonts %%%
\setmainfont[Ligatures=TeX]{Linux Libertine O}
\newfontfamily{\ezr}[Script=Hebrew]{EzraSIL}

\newfontfamily{\mainnolig}{Linux Libertine O}
\newcommand{\q}{{\mainnolig '}}


%%% Page layout %%%
\settypeblocksize{247mm}{160mm}{*}
\setlrmargins{*}{*}{1}
\setulmargins{*}{*}{1}
\checkandfixthelayout

%%% Hyperref (Information in PDF) %%%
\hypersetup{
unicode=true,
pdfauthor={Claus Tøndering},
pdftitle={Bible Online Learner: Users's Guide}
}

%%% Section numbering %%%
\setsecnumdepth{subparagraph}

%%% Lists %%%
\tightlists


%%% Colors %%%
\definecolor{dkgreen}{rgb}{0,0.6,0}
\definecolor{mauve}{rgb}{0.58,0,0.82}
\definecolor{shadecolor}{gray}{0.9}

%%% shaded environment %%%
\setlength{\FrameSep}{0.5\fboxsep}


%%% listings %%%
\lstset{frame=tb,
  aboveskip=3mm,
  belowskip=3mm,
  showstringspaces=false,
  keepspaces=true,
  columns=flexible,
  basicstyle={\footnotesize\ttfamily},
  numbers=none,
  numberstyle=\tiny\color{gray},
  keywordstyle=\color{blue},
  commentstyle=\color{dkgreen},
  stringstyle=\color{mauve},
  breaklines=true,
  breakatwhitespace=true,
  tabsize=3,
  escapechar=\%,
  captiondirection=LTR  % Required by bidi package (although its documentation says otherwise)
}

\renewcommand{\lstlistingname}{\textsc{Listing}}

\lstdefinelanguage{PHP}
{morekeywords={class,public,implements,function},
   morecomment=[l]//,
   morecomment=[s]{/*}{*/},
   morestring=[b]",
   morestring=[b]'
}

\lstdefinelanguage{TypeScript}
{morekeywords={class,interface,extends,string,number,boolean,any},
   morecomment=[l]//,
   morecomment=[s]{/*}{*/},
   morestring=[b]",
   morestring=[b]'
}

\lstdefinelanguage{CSS}
{
  moredelim={ [is][\color{blue}]{|}{|} },
  moredelim={ [is][\color{red}]{/}{/} }
}


%%% Chapter style %%%

% My own version of the ell chapter style:
\makechapterstyle{claus}{%
  \chapterstyle{default}
  \renewcommand*{\chapnumfont}{\normalfont\HUGE\sffamily}
  \renewcommand*{\chaptitlefont}{\normalfont\huge\sffamily}
  \settowidth{\chapindent}{\chapnumfont 111}
  \renewcommand*{\chapterheadstart}{\begingroup
    \vspace*{\beforechapskip}%
    \begin{adjustwidth}{}{-\chapindent}%
    \hrulefill
    \smash{\rule{0.4pt}{15mm}}
    \end{adjustwidth}\endgroup}
  \renewcommand*{\printchaptername}{}
  \renewcommand*{\chapternamenum}{}
  \renewcommand*{\printchapternum}{%
    \begin{adjustwidth}{}{-\chapindent}
    \hfill
    \raisebox{10mm}[0pt][0pt]{\chapnumfont\ifanappendix Appendix\else Chapter\fi\ \thechapter}%
                              \hspace*{1em}
    \end{adjustwidth}\vspace*{-3.0\onelineskip}}
  \renewcommand*{\printchaptertitle}[1]{%
    \vskip\onelineskip
    \raggedleft {\chaptitlefont ##1}\par\nobreak}}

% Default style should still use this font:
\renewcommand*{\chaptitlefont}{\normalfont\huge\sffamily}


%%% Indexing %%%

% Note: There is a problem with using \index in a (long)tabu* environment if the \index contains
% LaTeX commands, such as:
% \index{configuration JavaScript variable@\emph{configuration} JavaScript variable}
% Two things are required:
%    - The \emph must be preceded by \string
%    - The \index{...} must be replaced by \hmmindex{...} thus:
% \hmmindex{configuration JavaScript variable@\string\emph{configuration} JavaScript variable}}

\newcommand\hmmindex[1]{\index{#1}}

% Combines hyperref and italic page number in index
\newcommand*{\hyperit}[1]{\textit{\hyperpage{#1}}}

\renewcommand{\preindexhook}{Page numbers in italics are used to indicate the main references for an
  entry. A page number may appear both in italics and in upright letters if a term is used twice on
  a page.\label{sec-index}
\vspace{1cm}
}



%%% Auxiliary commands %%%
\newcommand*{\bibleref}[3]{#1~#2\thinspace:\thinspace#3}
\newcommand{\heb}[1]{{\RL {\ezr #1}}}
\newcommand{\forlater}[1]{} %Omit text
\newcommand*{\xml}[1]{\texttt{<#1>}}
\newcommand*{\xmla}[1]{\texttt{#1}} % An xml attribute


%%% Auxiliary tabu environments %%%
\tabulinesep=_2mm %Works together with the \addlinespace[...] values below

\newenvironment{my-longtabu}[2]{
%\begin{center}
\begin{longtabu*}{@{}#1@{}}
  \toprule
  #2\\\addlinespace[-1mm]
  \midrule
  \endhead

  \emph{\rmfamily\normalsize(Continued...)} & \\
  \endfoot

  \addlinespace[-1mm]\bottomrule
  \endlastfoot
}{%
\end{longtabu*}
%\end{center}%
}

\newcommand{\headii}[2]{\textbf{#1} & \textbf{#2}}
\newcommand{\headiii}[3]{\textbf{#1} & \textbf{#2} & \textbf{#3}}


% my-longtabu without \midrule
\newenvironment{my-longtabu-nomid}[2]{
\begin{center}
\begin{longtabu*}{@{}#1@{}}
  \toprule
  #2\\\addlinespace[-1mm]
  \midrule
  \endhead

  \emph{\rmfamily\normalsize(Continued...)} & \\
  & \\ % Required because the only instance of this table clashed with footnotes
  \endfoot

  \addlinespace[-1mm]\bottomrule
  \endlastfoot
}{%
\end{longtabu*}
\end{center}%
}

\newenvironment{my-tabu}[2]{%
\begin{center}
\begin{tabu}{@{}#1@{}}
  \toprule
  #2\\\addlinespace[-1mm]
  \midrule
}{%
\addlinespace[-1mm]\bottomrule
\end{tabu}
\end{center}%
}

%%% Allow extra space between words %%%
\sloppy

\captionnamefont{\small\itshape}
\captiontitlefont{\small\itshape}


\setfloatlocations{figure}{tb}

%% Command to select between to texts
\newcommand{\HOrG}[3]{\ifx#1h#2\else #3\fi}
\newcommand{\hebgr}{}   % Redefined in document
\newcommand{\hebgrlong}{}   % Redefined in document

%%% Front matter %%%
\title{Bible Online Learner:\\Technical Documentation}
\author{Claus Tøndering}
\date{11 July 2022}

\makeindex

\settocdepth{subsection}

\begin{document}
\begin{titlingpage*}
\maketitle

\begin{center}
Copyright © 2022 by Claus Tøndering, claus@tondering.dk

\vspace{5mm}

The document is made available under a Creative Commons Attribution 4.0 International License

(see \url{https://creativecommons.org/licenses/by/4.0/})
\end{center}
\end{titlingpage*}


\clearpage
\tableofcontents
\chapterstyle{claus} % TOC should be in default style. "claus" style starts here.

%%%%%%%%%%%%%%%%%%%%%%%%%%%%%%%%%%%%%%%%%%%%%%%%%%%%%%
%%%%%%%%%%%%%%%%%%%% Introduction %%%%%%%%%%%%%%%%%%%%
\chapter{Introduction}

\textbf{You should read this chapter.}
\plainbreak{3}

Bible Online Learner (or ``Bible~OL'' for short) is a web-based computer program that supports
reading and learning biblical Hebrew and Greek. Bible Online Learner provides these features:


\begin{itemize}
\item Users can read the Old Testament in Hebrew and the New Testament in Greek.
\item Users can see grammatical information about the words and clauses in the text.
\item Teachers of Hebrew and Greek can create exercises based on the biblical texts.
\item Students can drill and test themselves using exercises created by teachers.
\item Teachers can keep track of how well students do in the exercises.
\item Teachers can generate exams for students.
\end{itemize}


%%%%%%%%%%%%%%%%%%%% How To Read This Document %%%%%%%%%%%%%%%%%%%%
\section{How To Read This Document}

This documents describes how to use Bible Online Learner (Bible~OL). The document consists of three
parts:

\begin{itemize}
\item General information for everyone who wants to use Bible~OL.
\item Information specifically for students who want to use the exercises provided in Bible~OL.
\item Information specifically for teachers who want to create exercises in Bible~OL.
\end{itemize}

At the start of most chapters you will find a line indicating if that chapter is relevant for you.

On page \pageref{sec-index} you will find an index, which may help you find your way through this document.

%%%%%%%%%%%%%%%%%%%%%%%%%%%%%%%%%%%%%%%%%%%%%%%%%%%%%%%%%%%%%
%%%%%%%%%%%%%%%%%%%% User Interface %%%%%%%%%%%%%%%%%%%%
\chapter{User Interface}

\textbf{You should read this chapter.}
\plainbreak{3}

You access Bible~OL through a web browser using an appropriate
URL\footnote{Most users will use \url{https://bibleol.3bmoodle.dk}.}.

When you access the Bible~OL main web site using a computer, you will see an introductory page,
at the top of which there is a menu. If you are using a smartphone, you will see a small rectangle with
three horizontal lines at the top right of the screen. Tap that rectangle to display the menu.


The menu has five items:

\begin{itemize}
\item \emph{Home} -- Selecting this item, takes you to the main web page.
\item \emph{Text and Exercises} -- This allows you to view the Hebrew and Greek biblical texts and to run exercises.
\item \emph{User Access} -- Here you can log in to the system and view the privacy policy.
\item \emph{Language} -- This lets you select the language of the user interface.
\item \emph{Variant} -- Here you can select between different variants of the terms and translations used. (See Section XXX.)
\end{itemize}


If you have a user account (see Chapter \ref{chap-logging-in}), you can select Login from the User Access menu to
access your personal features of the system. Once you have logged, in, a new menu item appears:

\begin{itemize}
\item \emph{My Data} -- This menu item lets you change your user profile, setup font preferences, enroll in
  classes, and view how you are doing on the exercises.
\end{itemize}


%%%%%%%%%%%%%%%%%%%%%%%%%%%%%%%%%%%%%%%%%%%%%%%%%%%%%%
%%%%%%%%%%%%%%%%%%%% Viewing Text %%%%%%%%%%%%%%%%%%%%
\chapter{Viewing Text}\label{chap-viewing-text}

\textbf{You should read this chapter.}
\plainbreak{3}

The most basic operation of Bible~OL is that of viewing the Hebrew or Greek text of the Bible.
From the \emph{Text and Exercises} menu select \emph{Display text.} This will take you to a web page
displaying the dialog of Figure \ref{fig-selecttext}.
\begin{figure}
  \begin{center}
    \includegraphics[width=0.6\textwidth]{selecttext.png}
  \end{center}
  \caption{Selecting a text to display.}\label{fig-selecttext}
\end{figure}
Under ``Corpus'' in that dialog, you select the text database you want to use. You have three options:


\begin{itemize}
\item \emph{Hebrew (ETCBC4, OT):} This is the Hebrew text of the Old Testament as provided by the ETCBC4
  database (see Appendix XXX).
\item \emph{Hebrew (ETCBC4, Transliterated, OT):} This is the Hebrew text of the Old Testament as provided by
  the ETCBC4 database, but written with Latin letters.
\item \emph{Greek (Nestle 1904,NT):} This is the Greek text of the New Testament as provided by the Nestle
  1904 database (see Appendix XXX).
\end{itemize}

Section \ref{sec-heb-view-text} shows how to use Bible Online Learner to study a Hebrew text.
Section \ref{sec-gr-view-text} shows how to study a Greek text. (You only need to read one of these
sections.)

\newcommand{\viewexampleI}[1]{%
%
\renewcommand{\hebgr}{\HOrG{#1}{heb}{gr}}
\renewcommand{\hebgrlong}{\HOrG{#1}{Hebrew}{Greek}}
%
\section{Viewing a \hebgrlong{} Text}\label{sec-\hebgr-view-text}

To see \HOrG{#1}{\bibleref{Genesis}{1}{1-7}}{\bibleref{Luke}{2}{1-5}}, fill out the dialog in figure \ref{fig-selecttext} thus:

\begin{itemize}
\item Corpus: \HOrG{#1}{Hebrew (ETCBC4, OT)}{Greek (Nestle 1904, NT)}
\item Book: \HOrG{#1}{Genesis}{Luke}
\item Chapter: \HOrG{#1}{1}{2}
\item First verse: 1
\item Last verse: \HOrG{#1}{7}{5}
\item Show link icons: Don’t mark this. (See Section \ref{sec-resource-web}.)
\end{itemize}

Finally, click the \emph{Display} button. This will show you \HOrG{#1}{\bibleref{Genesis}{1}{1-7}}{\bibleref{Luke}{2}{1-5}} in \hebgrlong. (See Figure \ref{fig-\hebgr-text-a}.)
\begin{figure}
  \begin{center}
    \includegraphics[width=0.7\textwidth]{fig-\hebgr-text-a.png}
  \end{center}
  \caption{Displaying \HOrG{#1}{Genesis 1}{Luke 2}.}\label{fig-\hebgr-text-a}
\end{figure}

\HOrG{#1}{The small icon to the right of the first verse is a link to the same text at the SHEBANQ website.
(See Section \ref{sec-shebanq}.)}


\subsection{Viewing \hebgrlong{} Grammar Information}\label{sec-\hebgr-grammar-info}

If you are viewing the text on a computer, you will have three ways to display grammar information;
if you are using a tablet or a smartphone, you will have two ways to display grammar information.
They are:

\begin{enumerate}
  \item Hovering the mouse over a word or sentence part. (This is not available on tablets or
    smartphones.)
  \item Clicking a word or sentence part.
  \item Using the ``MyView'' selector.
\end{enumerate}

On a computer, you can use your mouse to point to a word in the text.\footnote{Known as letting your
  mouse ``hover'' over a word.} You will then see a so-called \emph{grammar information
  box.}\index{grammar information box|hyperit}
to the right of the text as in Figure \ref{fig-\hebgr-text-b}.
\begin{figure}
  \begin{center}
    \includegraphics[width=\textwidth]{fig-\hebgr-text-b.png}
  \end{center}
  \caption{Displaying the grammar information box on a computer.}\label{fig-\hebgr-text-b}
\end{figure}
In this box you will see detailed information about the word your mouse points to. When you move the
mouse, the grammar information box disappears. You may find this inconvenient, so instead you can
use the following method:

On a computer, table, or smartphone, you can click or tap on a word. In that case, a dialog box will
appear containing the grammar information box. (You can click the × at the top of the dialog box or
the Close button at the bottom of the box to close the dialog. Alternatively, press the ``Esc'' key
on your keyboard.)

A third way to display grammar information is to use the ``MyView'' selector as described in the
following section.

%%%%%%%%%%%%%%%%%%%% The Grammar Selection Box %%%%%%%%%%%%%%%%%%%%
\subsection{The ``MyView'' Selector}\label{sec-\hebgr-myview-selector}\index{MyView@``MyView'' selector}

Above the \hebgrlong{} text you see an ``eye'' labelled ``MyView''. If you click the eye icon labelled
\emph{grammar selection box}\index{grammar selection box}. At the same time the eye icon turns into
a × icon. The grammar selection box looks as shown in Figure \ref{fig-\hebgr-gram-sel-a}.

\begin{figure}
  \begin{center}
    \includegraphics[width=0.7\textwidth]{fig-\hebgr-gram-sel-a.png}
  \end{center}
  \caption{\hebgrlong{} grammar selection box.}\label{fig-\hebgr-gram-sel-a}
\end{figure}

The \hebgrlong{} grammar selection box contains four buttons, identifying the four levels of the
grammar hierarchy used by the \hebgrlong{} text: The text contains \emph{sentences,} which contain
\emph{\HOrG{#1}{clauses}{level 2 clauses},} which contain \emph{\HOrG{#1}{phrases}{level 1
    clauses},} which contain \emph{words.} You can click on each of these to display relevant
grammar information.

If, for example, you click the \emph{Word} button and then the \emph{Lexeme} button, the grammar
selection box looks as shown in Figure \ref{fig-\hebgr-gram-sel-b}.
\begin{figure}
  \begin{center}
    \includegraphics[width=0.7\textwidth]{fig-\hebgr-gram-sel-b.png}
  \end{center}
  \caption{\hebgrlong{} grammar selection box after clicking \emph{Word} and \emph{Lexeme}.}\label{fig-\hebgr-gram-sel-b}
\end{figure}
If you now click the \emph{Part of speech} button, the Hebrew text changes to look as in Figure
\ref{fig-\hebgr-text-c}, where you can see the part of speech of each word of the text.
\begin{figure}
  \begin{center}
    \includegraphics[width=0.7\textwidth]{fig-\hebgr-text-c.png}
  \end{center}
  \caption{\HOrG{#1}{Genesis 1}{Luke 2} with part-of-speech information.}\label{fig-\hebgr-text-c}
\end{figure}
You can add additional information by clicking the relevant buttons in the grammar selection box.

You can use the \emph{Clear grammar} all the selected grammar information, and you can use the ×
icon to hide the grammar selection box.

The grammar selection box also allows you to see borders between \HOrG{#1}{phrases, clauses,}{clauses} or sentences,
as well as grammatical information about each of these. If, for example, you click
\emph{\HOrG{#1}{Clause}{Clause level 1}}, and then select
\emph{Show border}, you will see the borders of each clause.

By pointing your mouse to the word ``\HOrG{#1}{Clause}{Clause1}'' on the border of a particular
Clause (or, alternatively, clicking the word ``\HOrG{#1}{Clause}{Clause1}''), a grammar information box
for the particular clause will be shown. (See Figure \ref{fig-\hebgr-text-d}.)

\begin{figure}
  \begin{center}
    \includegraphics[width=\textwidth]{fig-\hebgr-text-d.png}
  \end{center}
  \caption{Showing grammar information for the second clause of \HOrG{#1}{Genesis 1}{Luke 2}.}\label{fig-\hebgr-text-d}
\end{figure}

\HOrG{#1}{Sometimes clauses (or other parts of a sentence) can contain other clauses inside them. An example
  of this is seen in \bibleref{Genesis}{1}{7} (Figure \ref{fig-\hebgr-text-e}). Here, you can see how the
  clause \heb{וַיַּבְדֵּ֗ל בֵּ֤ין הַמַּ֨יִם֙ וּבֵ֣ין הַמַּ֔יִם} is split in two and contains the clause \heb{אֲשֶׁר֙ מִתַּ֣חַת לָרָקִ֔יעַ} inside it.
  The split clause is marked by its missing left and right borders.

\begin{figure}
  \begin{center}
    \includegraphics[width=0.6\textwidth]{fig-\hebgr-text-e.png}
  \end{center}
  \caption{A split clause in \bibleref{Genesis}{1}{7}.}\label{fig-\hebgr-text-e}
\end{figure}
}{Some words may not belong to a particular clause, and a clause may be split into parts. In Figure
  \ref{fig-gr-text-d} the word καὶ at the start of verse 3 is not a member of a clause; and the
  clause in verse 1, the clause is split into two pars around the word δὲ, which is not part of the clause.
  The split clause is marked by its missing left and right borders.}

The different items you can select in the grammar selection box are detailed in section XXX, but
\HOrG{#1}{a few items are}{one item is} worth mentioning here:

In the grammar selection box, under \emph{Word} and \emph{Lexeme} you can enter a ``Word frequency
color limit''. Setting this value to, for example, 50, means that the 50 most common
\HOrG{#1}{Hebrew or Aramaic}{Greek} words in the \HOrG{#1}{Old}{New} Testament will be displayed in
black, whereas rarer words will be display in blue. (See Figure \ref{fig-\hebgr-text-f}.) If you are
learning \hebgrlong{}, you may find this feature useful when deciding if a word is worth memorizing.
When determining how common words are, different morphological forms of the same word are counted as
one. For more information about word frequency, see section XXX.

\begin{figure}
  \begin{center}
    \includegraphics[width=0.7\textwidth]{fig-\hebgr-text-f.png}
  \end{center}
  \caption{The 50 most common words are black, rarer words are blue.}\label{fig-\hebgr-text-f}
\end{figure}

\HOrG{#1}{In most cases, the information you find in the grammar information box will be the same as
  what is shown between the lines using the ``MyView'' selector; but for glosses this is not the
  case. For example, in the grammar information box of Figure \ref{fig-\hebgr-text-b} you can see
  that the lexeme \heb{היה} is translated into English as ``be, happen, become, occur''. But if you
  open the ``MyView'' selector and choose \emph{Word} and \emph{Glosses} and \emph{English}, only
  the first gloss, ``be'', will be shown between the lines of Hebrew text. }{}

\HOrG{#1}{\subsection{Viewing a Transliterated Hebrew text}\label{sec-heb-translit}\index{transliterated
    Hebrew|see {Hebrew, transliterated}}\index{Hebrew!transliterated}

You can choose to view a Hebrew text in Latin letters rather Hebrew letters. You do this by
specifying the corpus ``Hebrew (ETCBC4, Transliterated, OT)'' in the dialog in figure
\ref{fig-selecttext}. The first two of Genesis~1 will look at show in Figure
\ref{fig-heb-translit}. You will notice that this text uses a number of variations of Latin letters plus
the special characters ʔ (not to be confused with a question mark) and ʕ. These two characters
correspond to the Hebrew characters \heb{א} and \heb{ע}, respectively.
\begin{figure}
  \begin{center}
    \includegraphics[width=0.7\textwidth]{fig-heb-translit.png}
  \end{center}
  \caption{\bibleref{Genesis}{1}{1-2} in transliterated Hebrew.}\label{fig-heb-translit}
\end{figure}}{}
}

\FloatBlock

\viewexampleI{h}

\FloatBlock

\viewexampleI{g}


%%%%%%%%%%%%%%%%%%%%%%%%%%%%%%%%%%%%%%%%%%%%%%%%%%%%
%%%%%%%%%%%%%%%%%%%% Logging In %%%%%%%%%%%%%%%%%%%%
\chapter{Logging In}\label{chap-logging-in}

\textbf{You should read this chapter.}
\plainbreak{3}

You can use Bible~OL to display biblical text and run some exercises without logging in to the
system. But if you want to customize the way text is displayed and if you want to take full
advantage of the system, you should have an account and log in when you use Bible~OL.

%%%%%%%%%%%%%%%%%%%% Creating An Account %%%%%%%%%%%%%%%%%%%%
\section{Creating An Account}\index{account!creating}\index{creating an account|see {account, creating}}

You can create an account yourself, or you can have your teacher create an account for you.

To create an account yourself, select \emph{Login} from the \emph{User Access} menu. This brings up
the login page, where you can select \emph{Create a new account}, or -- if allowed by your
installation -- you can use an existing Google or Facebook account to log in.


%%%%%%%%%%%%%%%%%%%% Advantages of Having An Account %%%%%%%%%%%%%%%%%%%%
\section{Advantages of Having An Account}

When you are logged in using an account, you get a number of extra possibilities:

\begin{itemize}
\item You can customize the fonts used for displaying text.
\item You can join classes set up by your teacher and access exercises that anonymous users cannot
  access.
\item When you take an exercise, your progress can be recorded and you and your teacher can monitor
  and grade your progress.
\end{itemize}

When you are logged in, an extra menu item appears on the Bible~OL website: ``My Data''. This menu
has these items:

\begin{itemize}
\item \emph{Font preferences} -- Use this to change the fonts used to display Hebrew or Greek.
\item \emph{Profile} -- Here you can change your name, e-mail address, preferred interface language,
  and password. (If you are logged in via Google or Facebook, you can only change your preferred
  interface language.)
\item \emph{Enroll/unenroll in class} -- Use this to join or leave a class. Classes are described in
  detail in Chapter XXX.
\item \emph{My performance} -- See how you are doing solving the exercises.
\end{itemize}


To log out, select \emph{Logout} from the \emph{User Access} menu.


%%%%%%%%%%%%%%%%%%%% Special Account Privileges %%%%%%%%%%%%%%%%%%%%
\section{Special Account Privileges}\index{account!privileges}\index{privileges|see {account, privileges}}

A system administrator may assign special privileges to your account. The special privileges are:

\begin{itemize}
\item Facilitator
\item Translator
\item Sysadmin
\end{itemize}

The facilitator and translator privileges are independent of each other. A user may have either or
both sets of privileges.

A sysadmin has all the privileges of both facilitators and translators plus additional privileges.



\subsection{Facilitator}\index{facilitator}\index{teacher}

As a facilitator (or teacher) you have the following additional rights on the system:

\begin{itemize}
\item Add, modify, or delete user accounts (except for facilitator, translator, or sysadmin accounts).
\item Create and manage exercises.
\item Create and manage exams.
\item Create and manage classes.
\item Monitor how the students in your classes are doing in the exercises.
\item Assign or remove facilitator privileges to other accounts.
\end{itemize}

When you are logged in with facilitator privileges, additional items appears in the ``My Data'' menu:

\begin{itemize}
\item \emph{Students' performance} -- See how your students are doing solving the exercises.
\item \emph{Grade Quizzes} -- Grade students' quizzes.
\item \emph{Grade Exams} -- Grade students' quizzes.
\end{itemize}

Additionally, you will see a new menu item ``Administration'' with these items:

\begin{itemize}
\item \emph{Users} -- Manage user accounts.
\item \emph{Classes} -- Manage classes.
\item \emph{Manage exercises.}
\item \emph{Manage exams.}
\end{itemize}


\subsection{Translator}\index{translator}

As a translator you have the right to modify the translation of

\begin{itemize}
\item The user interface.
\item The names for Hebrew and Greek grammatical terms.
\item The Hebrew and Greek lexicons.
\item Add new languages to the set of available languages.
\end{itemize}

When you are logged in with translator privileges, you will see a new menu item ``Administration'' with these items:

\begin{itemize}
\item \emph{Translate interface} -- Translate the user interface.
\item \emph{Translate grammar items} -- Translate the names for Hebrew and Greek grammatical terms.
\item \emph{Translate lexicon} -- Translate the Hebrew and Greek lexicons.
\item \emph{Download lexicon} -- Download a translation of a Hebrew or Greek lexicon.
\item \emph{Available translations} -- View the available translations and add new languages.
\end{itemize}

\subsection{Sysadmin}\index{sysadmin}\index{administrator}\index{system administrator}

As a sysadmin (or system administrator) you have all the privileges of facilitators and translators
plus these additional rights:

\begin{itemize}
\item Add, modify, or delete user accounts (including facilitator, translator, or sysadmin accounts).
\item Manage gloss links (see Section XXX).
\item Change ownership of exercises.
\item Manage exercises created by other facilitators.
\item Manage exams created by other facilitators.
\item Manage classes created by other facilitators.
\end{itemize}

The ``Administration'' menu will have one additional item:

\begin{itemize}
\item \emph{Gloss links} -- See Section XXX.
\end{itemize}


%%%%%%%%%%%%%%%%%%%%%%%%%%%%%%%%%%%%%%%%%%%%%%%%%%%%%%%%%%%
%%%%%%%%%%%%%%%%%%%% Running Exercises %%%%%%%%%%%%%%%%%%%%
\chapter{Running Exercises}\index{running exercises}\index{exercises!running}

\textbf{Read this chapter if you want to run or create exercises.}
\plainbreak{3}

In this section we will look at how you run an exercise. We will go through a couple of examples
that you can try out on the system.

NOTE: You need not be logged in to try the code in this example, but you may see some warnings if
you are not logged in.

Exercises are located in folders in much the same way files on a computer are located in folders.
You can access the exercises by selecting \emph{Exercises} from the \emph{Text and exercises} menu.
You will then see a list of folders that are found within the top level folder (Figure \ref{fig-top-level}).
\begin{figure}
  \begin{center}
    \includegraphics[width=0.4\textwidth]{toplevel.png}
  \end{center}
  \caption{The top level exercise folder.}\label{fig-top-level}
\end{figure}
The contents of some folders are available to all students, but you will see that some of the
folders here are marked ``Restricted''. These are folders that are only available to students that
are enrolled in certain classes. Students' access to folders is controlled by teachers as described
in section XXX.

Section \ref{sec-heb-exer} gives examples of how to run a Hewbrew exercise. Section
\ref{sec-gr-exer} gives examples of how to run a Greek exercise. (You only need to read one of these
sections.)


\newcommand{\exerciseexampleI}[1]{%
%
\renewcommand{\hebgr}{\HOrG{#1}{heb}{gr}}
\renewcommand{\hebgrlong}{\HOrG{#1}{Hebrew}{Greek}}
%

\FloatBlock

\section{\HOrG{#1}{Hebrew}{Greek} Exercises}\label{sec-\hebgr-exer}

\subsection{Example: First \HOrG{#1}{Hebrew}{Greek} Exercise}\label{sec-\hebgr-exer-i}

(For a corresponding \HOrG{#1}{Greek}{Hebrew} exercise, see Section \HOrG{#1}{\ref{sec-gr-exer-i}}{\ref{sec-heb-exer-i}}.)

Find the folder named ``\HOrG{#1}{ETCBC4}{Nestle~1904}'' and click on that. Within that folder, you
will find another folder called ``demo''. If you click on that, you will see a list of all the
exercises found within that folder (Figure \ref{fig-\hebgr-folder}).
\begin{figure}
  \begin{center}
    \includegraphics[width=0.9\textwidth]{\hebgr-folder.png}
  \end{center}
  \caption{The \HOrG{#1}{ETCBC4}{Nestle~1904}/demo exercise folder.}\label{fig-\hebgr-folder}
\end{figure}

We will now focus on the exercise called ``demo1''. The person who created this exercise
will have configured it with a set of Bible passages that should be used for this exercises. If you
click on one of the numbers 5, 10, or 25 under the heading ``Select number of questions using preset
passages'', you will start an exercise with 5, 10, or 25 questions take from the pre-configured
Bible passages. If instead you click on one of the numbers 5, 10, or 25 under the heading ``Select
number of questions and specify your own passages'', you will be allowed to specify the Bible
passages yourself.

Click on the number 5 under ``...preset passages'' and the exercise will start. The system will show
you a random sentence from the specified Bible passages, for example, the one you see in Figure
\ref{fig-\hebgr-exer-i}.
\begin{figure}
  \begin{center}
    \includegraphics[width=0.7\textwidth]{fig-\hebgr-exer-i.png}
  \end{center}
  \caption{The first sentence of a \hebgrlong{} exercise.}\label{fig-\hebgr-exer-i}
\end{figure}
At the top of the exercise you see a short description -- in this case ``Please indicate
  the gender and number of these nouns and pronouns.'' If you click the button labelled
``Locate''\index{locate}, you will learn that this sentence is found in
\HOrG{#1}{\bibleref{Genesis}{1}{6}}{\bibleref{Matthew}{8}{26}}. As described in Section
\ref{sec-\hebgr-grammar-info}, you can see more information about each word by pointing to it with
your mouse (if you are using an ordinary computer), by clicking on it, or by using the ``MyView''
selector.

  In this example, the system has highlighted three nouns or pronouns in purple. Your task is to identify the
grammatical gender and number of each of these words.

Below the text you see a green bar. This will show your progress through the five sentences of this
exercise. Below the bar is a box in which you should provide your answers. This answer box provides
some information about the noun: In this case it its the actual text and the English translation of
the word. These items are known as the ``Display features''\index{display feature} (see Figure \ref{fig-\hebgr-exer-ia}).
Below the display features is the information that you are expected to provide: In this case it is
the gender and number of the word. These items are known as the ``Request features''\index{request feature}.

\begin{figure}
  \begin{center}
    \includegraphics[width=0.7\textwidth]{fig-\hebgr-exer-ia.png}
  \end{center}
  \caption{Display features and request features.}\label{fig-\hebgr-exer-ia}
\end{figure}

The first noun, \HOrG{#1}{\heb{רָקִ֖יעַ}}{ἀνέμοις}, is shown in the top line of the answer box, and you
must identify its gender by clicking either \HOrG{#1}{``Masculine'', ``Feminine'', or ``Other
  value'',\footnote{``Other value'' is not relevant in this exercise.}}{``N/A'',\footnote{That is,
    ``Not applicable''. This means that the word has no gender, which is the case for pronouns such
    as ``I'' and ``you''.} ``Masculine'', ``Feminine'', or ``Neuter''.} and its
number by clicking either \HOrG{#1}{``Singular'', ``Plural'', ``Dual'', or ``Other
  value''.}{``N/A'', ``Singular'', or ``Plural''.}

After making your choices, you may then check your answer by clicking ``Check answer''. Figure
\ref{fig-ans-\hebgr-i} shows you what a correct and a wrong answer looks like.
\begin{figure}
  \begin{center}
    \includegraphics[width=0.7\textwidth]{fig-ans-\hebgr-i.png}
  \end{center}
  \caption{A correct and a wrong answer to a \hebgrlong{} exercise.}\label{fig-ans-\hebgr-i}
\end{figure}
If you don't know the answer, clicking the ``Show answer'' button will display the correct answer.

In this sentence there are three nouns. We can move on to the next noun,
\HOrG{#1}{\heb{תֹ֣וךְ}}{θαλάσσῃ}, by clicking the \raisebox{-0.1\baselineskip}{\includegraphics[height=9pt]{right-angle.png}} symbol at the right side of the answer box.
A \raisebox{-0.1\baselineskip}{\includegraphics[height=9pt]{left-angle.png}} symbol will then appear at the left side of the answer box, allowing you to move back
to the previous word.

When the last word in the sentence, \HOrG{#1}{\heb{מָּ֑יִם}}{γαλήνη}, is shown (Figure \ref{fig-\hebgr-exer-ii}), three
buttons appear below the answer box.
\begin{figure}
  \begin{center}
    \includegraphics[width=0.7\textwidth]{fig-\hebgr-exer-ii.png}
  \end{center}
  \caption{Showing the last word of the exercise question.}\label{fig-\hebgr-exer-ii}
\end{figure}
Use the ``Next'' button to move to the next sentence. When there are no more sentences, or if you
want to terminate the exercise prematurely, you can click either ``GRADE task'' or ``SAVE outcome''.
Both of these terminate the exercise. If you are not logged in, no further action is taken,
regardless of which of the two buttons you press. But if you \emph{are} logged in, ``GRADE task''
causes the system to stores your result internally and indicates to the teacher that your result may
be used for grading your progress. ``SAVE outcome'' also stores your result internally, but
indicates that you do not want your result to be used for grading.

\FloatBlock

\subsection{Example: Second \HOrG{#1}{Hebrew}{Greek} Exercise}\label{sec-\hebgr-exer-ii}

(For a corresponding \HOrG{#1}{Greek}{Hebrew} exercise, see Section \HOrG{#1}{\ref{sec-gr-exer-ii}}{\ref{sec-heb-exer-ii}}.)

We will now look at the exercise called ``demo2'' in the
``\HOrG{#1}{ETCBC4/demo}{Nestle~1904/demo}'' folder. Click on the number 5 under ``...preset
passages'' for that exercise. The system will show you a random sentence from the specified Bible
passages, for example, the one you see in Figure \ref{fig-\hebgr-exer-iii}.

\begin{figure}
  \begin{center}
    \includegraphics[width=0.7\textwidth]{fig-\hebgr-exer-iii.png}
  \end{center}
  \caption{A sentence from the exercise ``\HOrG{#1}{ETCBC4/demo2}{Nestle 1904/demo2}''.}\label{fig-\hebgr-exer-iii}
\end{figure}


Here, we have a sentence from \HOrG{#1}{\bibleref{Genesis}{3}{14}}{\bibleref{Luke}{8}{17}}. You will
immediately notice an important difference from the previous example: Some of the words have been
replaced by a number in parentheses. The reason is that in this case Bible Online Learner asks the user to
provide a word that is actually part of the text; therefore the word must not be shown in the
window. The system has therefore replaced the interesting words with numbers.

In this example, the answer box contains five lines labelled \HOrG{#1}{``Item number'', ``Lexeme
  (with variant)'', ``Gender'', ``Number'', and ``State''}{``Item number'', ``Lexeme'', ``Tense'',
  ``Mood'', ``Voice'', ``Person'', and ``Number''}. The item number refers to the number in
parentheses in the sentence. \HOrG{#1}{In this case we have moved through the words to the third
  word.}{} The lexeme is the dictionary form of the word in question, and the \HOrG{#1}{number,
  gender, and state}{voice, mood, person, number, and tense} should help you create the word form
that is actually in the text.

Your task is to type the word form that is found in the text\HOrG{#1}{, but without the Hebrew
  cantillation marks (which you rarely need to know in detail).}{ in lower case without
  accents.\footnote{The label ``Normalized'' refers to a version of the text without punctuation and
    certain accents.}} Below the empty field
  for the text, you will see a few buttons labelled with \hebgrlong{} characters. You can use these
  buttons to spell the correct word form, which in this case is \HOrG{#1}{\heb{יְמֵי}}{εστιν}. The key
  marked \HOrG{#1}{$\rightarrow$}{$\leftarrow$} is a backspace key that deletes the last character
  you entered. The small characters in the upper \HOrG{#1}{right}{left} corner can be used to type
  the \hebgrlong{} character on you computer keyboard, if you prefer to do so rather than to click
  with your mouse. (More about this in Section XXX.)

When you have entered your answer, you can use the ``Check answer'' button to verify that your
answer is correct.

\FloatBlock


\subsection{Example: Third \HOrG{#1}{Hebrew}{Greek} Exercise}\label{sec-\hebgr-exer-iii}

(For a corresponding \HOrG{#1}{Greek}{Hebrew} exercise, see Section \HOrG{#1}{\ref{sec-gr-exer-iii}}{\ref{sec-heb-exer-iii}}.)

We will now look at the exercise called ``demo3'' in the
``\HOrG{#1}{ETCBC4/demo}{Nestle~1904/demo}'' folder. Click on the number 5 under ``...preset
passages'' for that exercise. The system will show you a random sentence from the specified Bible
passages, for example, the one you see in Figure \ref{fig-\hebgr-exer-iv}.

\begin{figure}
  \begin{center}
    \includegraphics[width=0.7\textwidth]{fig-\hebgr-exer-iv.png}
  \end{center}
  \caption{An exercise about clauses.}\label{fig-\hebgr-exer-iv}
\end{figure}

In this exercise the object of the exercise is not words but clauses. Each subquestion presents a
\HOrG{#1}{clause}{level 1 clause}, and your task is to identify the \HOrG{#1}{type}{function} of the
clause. Figure \ref{fig-\hebgr-exer-iv} shows a sentence from
\HOrG{#1}{\bibleref{Genesis}{20}{3}}{\bibleref{Matthew}{6}{11}}. If you click the ``MyView'' icon
and select \HOrG{#1}{\emph{Clause}}{\emph{Clause level 1}} and \emph{Show border}, the limits of
each clause are obvious. (See Figure \ref{fig-\hebgr-exer-v}.)

\begin{figure}
  \begin{center}
    \includegraphics[width=0.7\textwidth]{fig-\hebgr-exer-v.png}
  \end{center}
  \caption{Showing the limits of each clause.}\label{fig-\hebgr-exer-v}
\end{figure}

In this example you are asked to consider the \HOrG{#1}{type}{function} of the indicated clause. The
first clause is \HOrG{#1}{\heb{הִנְּךָ֥ מֵת֙ עַל־הָאִשָּׁ֣ה}}{Τὸν ἄρτον ἡμῶν τὸν ἐπιούσιον}, and you must decide
if the \HOrG{#1}{type}{function} of this clause is either \HOrG{#1}{\emph{AjCl} (adjectival clause),
  \emph{NmCl} (nominal clause), or \emph{Ptcp} (participle clause).}{\emph{object, indirect object,}
  or \emph{second object.}}\footnote{The answer \emph{Other value} is not relevant in this
  exercise.} Once you have made your choice and checked if it is correct, you can press the
\raisebox{-0.1\baselineskip}{\includegraphics[height=9pt]{right-angle.png}} symbol to move on to the
next clause in the sentence, which is \HOrG{#1}{}{simply the word} \HOrG{#1}{\heb{וְהִ֖וא בְּעֻ֥לַת
    בָּֽעַל}}{ἡμῖν}. \HOrG{#1}{(The middle clause in the sentence (\heb{אֲשֶׁר־לָקַ֔חְתָּ}) has a type that is not
  covered by this exercise, and for that reason it is not shown in purple and is omitted from the
  exercise.)}{(The middle clause in the sentence (δὸς) and the last one (σήμερον) have functions
  that are not covered by this exercise, and for that reason they are not shown in purple and are
  omitted from the exercise.)}

}

\exerciseexampleI{h}
\exerciseexampleI{g}

\FloatBlock


\section{Variations to Exercises}\index{variations to exercises}\index{exercises!variations}

The exercises presented in the previous sections illustrate how most exercises work. However, a
teacher can vary certain details about the way exercises look to the student. These details are
listed in the following sections.

\subsection{Fixed Exercises}\index{order of exercises}\index{exercises!fixed}\index{exercises!order}

Normally, the sentences for exercises are chosen at random from a set of Bible passages. This means
that every time you run an exercise, you may see new sentences.

However, an exercise may be constructed to display a fixed set of questions in a fixed order. If
that is the case, you will always see the same sentences every time you run the exercise.

Furthermore, the student can normally choose between seeing 5, 10, or 25 questions in an exercise;
but a teach may restrict this number so that only a fixed number of questions can be shown.

If either the order of questions or the number of questions is fixed, the students cannot themselves
choose the Bible passages for the exercise.

\subsection{Sentence Context}\index{sentence context}\index{context}

Normally, an exercise will show you one sentence at a time. However, an exercise may be constructed
to display a few sentences surrounding the sentence in question. For example, Figure
\ref{fig-context} shows one sentence in grey before and after the sentence that the exercise is
concerned with.

\begin{figure}
  \begin{center}
    \includegraphics[width=0.7\textwidth]{fig-context.png}
  \end{center}
  \caption{Context around the sentence being considered is shown in grey. The sentence in question
    in shown in black and purple.}\label{fig-context}
\end{figure}


\subsection{Disabling ``Locate''}\index{disabling locate}\index{locate}

A teacher may disable the ``Locate'' button in an exercise.


\subsection{Hints}\index{hints}

Sometimes a Hebrew word form, taken on its own, may have multiple interpretations. For example, the
word form \heb{תֶחֱזֶה} can be both 2nd person masculine and 3rd person feminine. A teacher may
configure an exercise to provide hints to the correct interpretation. Figure \ref{fig-ambig} shows
a sentence where the student is asked to provide the gender for this particular word. A hint tells
the student that we are dealing with a 2nd person form, which aids the student in selecting the
correct gender, masculine.

\begin{figure}
  \begin{center}
    \includegraphics[width=0.7\textwidth]{fig-ambig.png}
  \end{center}
  \caption{An ambiguous word form with a hint.}\label{fig-ambig}
\end{figure}


\subsection{Hidden Information}

When viewing text outside an exercise, you have access to a considerable amount of grammatical
information. When doing an exercise, some of that information may be inaccessible, either because
the information would give away the correct answer, or because the teacher has deliberately hidden
some information.

As an example, consider the Hebrew exercise \emph{demo1} presented in section \ref{sec-heb-exer-i}, in
which you are required to provide the gender of a Hebrew noun. Gender information in normally
available via the ``MyView'' selector, but as Figure \ref{fig-no-gender} shows, the ``person,
gender, number'' button has been disabled. Also, hovering the mouse over a word or clicking a word, will
not display gender information.

\begin{figure}
  \begin{center}
    \includegraphics[width=0.7\textwidth]{fig-no-gender.png}
  \end{center}
  \caption{``Person, gender, number'' disabled.}\label{fig-no-gender}
\end{figure}


\subsection{Typing Nothing}

Occasionally you come across a question whose answer is \emph{nothing}.

Imagine, for example, a quiz about the English language in which you are required to type the plural
ending of various words. What is the plural ending of the word ``cow''? It is, of course, ``s'' since
the plural of ``cow'' is ``cows''. But what is the plural ending of the word ``sheep''? Since the plural
of ``sheep'' is ``sheep'', the plural ending is ``~'' -- nothing!

How do you type nothing? If you leave an input field empty, Bible~OL will think that you have not
yet answered the question. The correct way to indicate that an answer is an empty text is to type a
single dash (hyphen, minus) ``-'' in the answer field.

As an example, consider question in Figure \ref{fig-empty-answer}. Here, Bible Ol asks for the
pronominal suffix of the word \heb{כְבֹ֥וד}; but that word has no pronominal suffix, so the correct
answer is an empty text, which you indicate by typing a dash by pressing the character button
indicated by the red arrow in the illustration.

\begin{figure}
  \begin{center}
    \includegraphics[width=0.7\textwidth]{fig-empty-answer.png}
  \end{center}
  \caption{When the answer to a question is an empty text, type a dash.}\label{fig-empty-answer}
\end{figure}


%%%%%%%%%%%%%%%%%%%%%%%%%%%%%%%%%%%%%%%%%%%%%%%%%%%%%%%%%%%%%%%%
%%%%%%%%%%%%%%%%%%%% Extra chapters %%%%%%%%%%%%%%%%%%%%

\chapter{Managing Exercises}\index{exercises!managing}\index{managing exercises|see {exercises, managing}}

\textbf{Read this chapter if you are a teacher who needs to create and manage exercises for students.}
\plainbreak{3}

Note: In order to manage exercises, you must be logged in with a username that has ``facilitator''
privileges.


\section{Creating Exercises}\index{exercises!creating}\index{creating exercises|see {exercises, creating}}

Before we get into the details of how to create exercises, we need to discuss two important concepts
used by Bible~OL:

\begin{description}
\item[Sentence units] A sentence can be seen as consisting of a set of \emph{sentence units}. By far
  the most common thing is to see a sentence as a set of words, but you could also see the sentence
  as comprised of clauses or phrases. Thus a sentence unit can be a \emph{word}, a \emph{clause}, a
  \emph{phrase}, or perhaps something else. As you read on, you will rarely go wrong if you assume
  that ``sentence unit'' means ``word''.
\item[Features] A sentence unit has various \emph{features.} A feature has a name and a value. For
  example, a word can have a feature called \emph{part of speech} with the value \emph{noun} and a
  feature called \emph{gender} with the value \emph{masculine.} An important feature is called \emph{text,}
  which simply refers to the actual characters making up the word; for example, the \emph{text} feature may
  have the value \emph{``elephant''.}
\end{description}

You should also realise that in the context of Bible~OL, an exercise is actually a description of
how the program should generate questions. Internally, an exercise is stored in a file whose
filename ends with ``.3et''.

An exercise specifies:

\begin{itemize}
\item The database that is to be used (typically, the Old or the New Testament).
\item The Bible passages from which the program chooses the exercise sentences (for example, the
  minor prophets).
\item The criteria that the program should use when choosing sentences.
\item The criteria that the program should use when choosing the sentence units (typically, words)
  that form the actual questions.
\item The sentence unit features whose values are shown to the user.
\item The sentence unit features whose values are requested from the user.
\end{itemize}

In the following examples, we shall create a few exercises. In order to do this, you must be logged
in with a username that has ``facilitator'' privileges.


\newcommand{\exerciseexampleII}[1]{%
%
\renewcommand{\hebgr}{\HOrG{#1}{heb}{gr}}
\renewcommand{\hebgrlong}{\HOrG{#1}{Hebrew}{Greek}}
%
\section{Creating a Simple \hebgrlong{} Exercise}\label{sec-create-simple-\hebgr-exercise}

We shall create an exercise in the conjugation of \hebgrlong{} verbs in the \HOrG{#1}{qal perfect
  forms}{present tense}. We shall ask the user to identify the \HOrG{#1}{person, gender, and number of
the perfect forms of various qal verbs.}{person and number of various present tense verb forms.}

From the \emph{Administration} menu select \emph{Manage exercises} then navigate to a folder where
you want to create your exercises. This should preferably be a folder used only by you and your
team. At the bottom of the page, click the \emph{Create exercise} button. A dialog will appear in
which you select the text database on which you want to base your exercise. Here, you should select
``\HOrG{#1}{Hebrew (ETCBC4, OT)}{Greek (Nestle 1904, NT)}'' and press the \emph{OK} button.

You will now se a web page that looks like Figure \ref{fig-\hebgr-create-i}. At the top you'll see
five tabs, labelled ``Description'', ``Passages'', ``Sentences'', ``Sentence Units'', and
``Features''. On the left, you can see the name of the text database you are using.
\begin{figure}
  \begin{center}
    \includegraphics[width=0.7\textwidth]{fig-\hebgr-create-i.png}
  \end{center}
  \caption{Initial window when creating an exercise.}\label{fig-\hebgr-create-i}
\end{figure}


The ``Description'' tab is currently displayed, and below it you see a text editing field in which
you can write information and instructions to the students who will be running this exercise. You
may, for example, write something like what you see in Figure \ref{fig-\hebgr-create-ii}.
\begin{figure}
  \begin{center}
    \includegraphics[width=0.7\textwidth]{fig-\hebgr-create-ii.png}
  \end{center}
  \caption{Filling out the ``Description'' field.}\label{fig-\hebgr-create-ii}
\end{figure}
}

\exerciseexampleII{h}
\exerciseexampleII{g}

\FloatBlock




\chapter{Viewing Statistics}
\chapter{User and Class Management}
\chapter{Creating Translations}
\chapter{Exams}
\chapter{User Preferences} % Profile and font prefs.
\chapter{Variants}





%%%%%%%%%%%%%%%%%%%%%%%%%%%%%%%%%%%%%%%%%%%%%%%%%%%%%%%%%%%%%%%%
%%%%%%%%%%%%%%%%%%%% Complementary Websites %%%%%%%%%%%%%%%%%%%%
\chapter{Complementary Websites}

\textbf{Read this chapter if you want to.}
\plainbreak{3}


A few additional websites complement the function of Bible~OL: The resource website and the SHEBANQ
website.



%%%%%%%%%%%%%%%%%%%% The Resource Website %%%%%%%%%%%%%%%%%%%%
\section{The Resource Website}\label{sec-resource-web}\index{resource website|hyperit}

The resource web site is a collection of photos from the Middle East. Many of them relate to events
and places described in the Bible. The photos have descriptive texts that contain Bible references.
The URL of the resource website is \url{https://resources.3bmoodle.dk}.

Bible~OL can use information from the resource website to add picture links to Bible passages. If a
photo in the resource website refers to, for example, \bibleref{Exodus}{3}{2}, and a user ticks the
``Show link icons'' checkbox when displaying Exodus chapter 3, a green ``P'' icon will appear in the
text next to verse 2 (see Figure \ref{fig-exodusiii}). Clicking on the icon will cause the web
browser to display the relevant photo. If there are more than one photo, the icon will be blue
rather than green.


\begin{figure}
  \begin{center}
    \includegraphics[width=0.7\textwidth]{exodus3.png}
  \end{center}
  \caption{The green P icon is a hyperlink to a photo that is relevant to \bibleref{Exodus}{3}{2}.}\label{fig-exodusiii}
\end{figure}


\begin{comment}

TODO:

In addition to the pictures, the resource website may also provide URLs associated with various
Bible verses. These URLs are configured in the resource website but are otherwise unrelated to the
functioning of that website. The URLs are intended to identify videos, documents, or other resources
that may relevant for studying a particular verse.

Information about these URLs is also retrieved by the cron job above. By accessing the URL
\url{https://resources.3bmoodle.dk/jsonurls.php}, the \emph{Ctrl\_pic2db} controller receives a JSON
object from the resources website containing information about the URLs and the Bible verses to
which they refer. Bible~OL stores this information in the \emph{bible\_urls}%
\index{bible\_urls table}
table in the user database\index{user database} (see Section \ref{sec-bible-urls}). Links to these
urls are displayed as ``V'', ``D'', or ``U'' icons.

\end{comment}


%%%%%%%%%%%%%%%%%%%% The SHEBANQ Website %%%%%%%%%%%%%%%%%%%%
\section{The SHEBANQ Website}\label{sec-shebanq}\index{SHEBANQ}

SHEBANQ (System for HEBrew text: ANnotations for Queries and markup) is a website that uses the
ETCBC4 database for displaying text and grammar information for the Hebrew Bible. The URL is
\url{https://shebanq.ancient-data.org}.

When Bible~OL displays a text from the Old Testament, an icon in the upper right corner of the text
area provides a link to the same chapter at the SHEBANQ website. A similar link to Bible~OL is found
on the SHEBANQ website. Also, when a teacher is creating an exercise in Bible~OL, they can import
MQL queries from the SHEBANQ website (see Section XXX \textbackslash{}ref\{sec-shebanq-import\}).



%%%%%%%%%%%%%%%%%%%%%%%%%%%%%%%%%%%%%%%%%%%%%%%%%%%%%%%%%%%%%%%%%%
%%%%%%%%%%%%%%%%%%%% Appendix: ETCBC4 Details %%%%%%%%%%%%%%%%%%%%
\appendix
\chapter{ETCBC4 Details}\label{app-etcbc}\index{ETCBC4|hyperit}

The \emph{ETCBC4} Emdros database contains the Hebrew and Aramaic text for the Old
Testament.\index{Old Testament}

The database comes from the \emph{Eep Talstra Center for Bible and Computer}%
\index{Eep Talstra Center for Bible and Computer}
and is made available under a Creative Commons Attribution-NonCommercial 4.0 International
License\index{license}.\footnote{\url{https://creativecommons.org/licenses/by-nc/4.0}.} When
describing the database, a text similar to this one should be used: ``The database itself can be
found through this persistent identifier: \texttt{urn:nbn:nl:ui:13-048i-71}.'' The identifier should
be a hyperlink pointing to
\url{https://www.persistent-identifier.nl/?identifier=urn:nbn:nl:ui:13-048i-71}.

Prior to using this database in Bible~OL, I have added additional features to the information from
its original creators. More details about this is given in Section XXX \textbackslash{}ref\{etcbc-origin\}.


%%%%%%%%%%%%%%%%%%%%%%%%%%%%%%%%%%%%%%%%%%%%%%%%%%%%%%%%%%%%%%%%%%%%%%
%%%%%%%%%%%%%%%%%%%% Appendix: Nestle1904 Details %%%%%%%%%%%%%%%%%%%%
\chapter{Nestle1904 Details}\label{app-nestle}\index{nestle1904|hyperit}

The \emph{nestle1904} database is in the public domain\index{license} and derives from the 1904
version of Nestle's Greek New Testament\index{New Testament} text.





%%%%%%%%%%%%%%%%%%%%%%%%%%%%%%%%%%%%%%%%%%%%%%%
%%%%%%%%%%%%%%%%%%%% Index %%%%%%%%%%%%%%%%%%%%
\printindex

\end{document}

% Local Variables:
% mode: latex
% ispell-dictionary: "british-ize"
% ispell-extra-args: ("--home-dir=/home/claus/Projects/BibleOL/usersguide")
% eval: (auto-fill-mode 1)
% End:
